\section{Outlook}

The experimental research completed to investigate the suitability of AMs as a lower-costing alternative for multi-channel detection of biomagnetic fields of the human brain are discussed in Sec. \ref{sec:level1}. The results provide preliminary promising results where the AM response to sensory, auditory, spontaneous and motive stimuli are deemed to be within an acceptable agreement with the commerical SQUID detection system. Additionally, in Ref. [\citen{XiaMagnetoencephalographyMagnetometer}] the human sized magnetic shielding developed illustrates the improvement of using AMs compared to SQUIDs. The measured sensitivity of the AM in Ref. [\citen{Shah2013AApplications}] is comparable to the operation of commercial SQUID devices. However, due to the limited bandwidth the AM operation they may not be suitable of measurement of all applications of MEG. 

The progress made in Ref. [\citen{Boto2017AMagnetometers}] constructing a 3D printed head-cast to host the AM sensing devices further illustrates benefits of operation at non-cryogenic temperatures. Therefore the AMs can be attached to the head-cast such that there is a 6.5 mm distance between the center of the sensing area and the patients skin. The benefit is an increased S/N and detection of deeper magnetic fields. Additionally, further advantages include the possibility of providing patient specific head-casts, similarly to providing patient head-molds for radiotherapy treatment, and the flexibility in AM positioning. The next stage would be to fabricate more AMs and complete multiple-channel detection to investigate correlations with the 13 channel detection completed using one AM. The rf AM used for MIT shows promising initial results for future implementation to detect and diagnose abnormalities in biological tissue detected from atypical tissue conductivity.   
  
  



