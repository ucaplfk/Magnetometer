\section{Introduction}
Magnetic field sensing has a vast variation in applications across a number of fields including: navigation, geophysics, archeology and diagnostic medicine \citep{Budker2007OpticalMagnetometry,Khedr2017AApplications}. Biomagnetic sensing is an active area of research where the development of diagnostic techniques such as: magnetoencephalography (MEG), magnetocardiography (MCG), magnetic resonance imaging (MRI) and magnetic induction sensing (MIT) aid in diagnosing and treating many medical conditions. MEG provides a noninvasive method to detect brain activity through detection of the very weak < 1 pT magnetic field response. Commercial MEG systems currently consist of $\approx$ 200 superconducting quantum interference devices (SQUIDs) in an array \citep{Sander2012MagnetoencephalographyMagnetometer.}. However, it is extremely expensive to build and operate these MEG systems due to the requirement for cryogenic cooling \citep{Knappe2014Optically-PumpedMEG}. It is critical that a more affordable alternative option is developed such that continuous measurements of brain responses expands our understanding of typical and atypical brain operation. Therefore, more accessible MEG systems would enable clinical trails and research into treatment of patients with mental health illness such as schizophrenia, dementia, depression, and epilepsy \citep{Johnson2010MagnetoencephalographyMagnetometer}.

NIST is a leader in pioneering chip-scale sensing device development. The first atomic clock developed in 2004 used microelectro-mechanical systems (MEMS) technology could also be operated as an atomic magnetometer (AM). The sensitivity of 40 fTHz$^{-1/2}$ was achieved using coherent population trapping as the pumping technique \citep{Schwindt2004Chip-scaleMagnetometer}. Improved operation schemes and development of the spin-exchange relaxation (SERF) regime enables measurement of biomedical fields with a significant increase in sensitivity \citep{Allred2002High-SensitivityRelaxation}. AM development is currently under investigation as a replacement for SQUID MEG systems. The AM operational sensitivity is approaching the 3–4 fTHz$^{-1/2}$ sensitivity of commercial SQUID MEG systems \citep{Shah2013AApplications}. The key advantage of employing AM arrays is that they do not require cryogenic cooling. Consequently fabrication is more cost-efficient and the sensors can be placed closer to the target \citep{Boto2017AMagnetometers}.  Additionally, AMs also show potential for diagnosing atrial fibrillation by completing MIT \citep{Deans2016OpticalHeart}.  

